\documentclass{article}
\usepackage[backend=bibtex,defernumbers=true]{biblatex}

\addbibresource{override.bib}
\addbibresource{generated.bib}
\addbibresource{curated.bib}

\defbibfilter{bvoredg}{keyword=bv or keyword=edg}
\defbibcheck{old}{\ifnumless{\thefield{year}}{2018}{}{\skipentry}}
\defbibcheck{new}{\ifnumgreater{\thefield{year}}{2018}{}{\skipentry}}

\begin{document}

Here we use \verb|\nocite{*}| to cause all entries in the bib files to be available for listing.  In normal documents one would not do this and instead let the various \verb|\site{...}| calls determine the possible contents of bibliographies.

\nocite{*}
\printbibliography[keyword={bv},title={BV}]
\printbibliography[keyword={edg},title={EDG}]
\printbibliography[notkeyword={bv},title={NOT BV}]
\printbibliography[notkeyword={edg},title={NOT EDG}]
\printbibliography[keyword={bv},keyword={edg},title={BV AND EDG}]
\printbibliography[filter=bvoredg,title={BV OR EDG}]
\printbibliography[check=new,title={NEW}]
\printbibliography[check=old,title={OLD}]

\printbibliography[keyword={microboone},title={MicroBooNE}]
\printbibliography[keyword={dayabay},title={Daya Bay}]
\end{document}
